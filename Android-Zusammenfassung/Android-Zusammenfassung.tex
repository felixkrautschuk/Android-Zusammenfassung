%gibt an: Papierformat, einseitiger Druck, Schriftgr��e
\documentclass[a4paper,oneside,titlepage,12pt]{article}
%-------------------------------------------------------------------
\usepackage[a4paper, top=2cm, footskip=0pt, headheight=0.8cm, headsep=0.6cm, lmargin=3cm, rmargin=2cm]{geometry}
\usepackage{graphicx}
\usepackage{helvet}
\usepackage{amsmath}
\usepackage{amsthm}
\usepackage{amssymb}
\usepackage{hyperref} 
\usepackage[right]{eurosym}
\usepackage[latin1]{inputenc}

%--------------------------------------------------------------------
\renewcommand{\baselinestretch}{1.2}

\begin{document}
%--------------------------------------------------------------------
%Titelseite
\begin{titlepage}
	\includegraphics{grafiken/HTW-Logo.png}
	%\includegraphics[width=.3\textwidth]{grafiken/HTW-Logo.png}
	\vspace*{3cm}
	\begin{center}
		\Huge{Android-Zusammenfassung\\} \vspace*{1cm}
		\huge{Felix Krautschuk\\}
		\small{(Matrikelnummer: 34230)}
		\vspace*{2cm}
		\normalsize{
			\\Studiengang Informatik\\
		}
	\end{center}
	\vspace{2cm}
\begin{center}
\large{ 5. Semester }
\end{center}
	\vspace*{3cm}



\end{titlepage}

\thispagestyle{empty}\clearpage

%----------------------------------------------------------------------------------------------------------------
%\rmfamily \pagestyle{fancy} \setcounter{secnumdepth}{4}
\newtheorem{satz}{Satz}
\newtheorem{lemma}[satz]{Lemma}
\newtheorem{folgerung}[satz]{Folgerung}
\theoremstyle{definition}
\newtheorem{definition}[satz]{Definition}
\numberwithin{equation}{section}
\renewcommand{\proofname}{Beweis}

\pagenumbering{roman}\setcounter{page}{3} \tableofcontents
\newcounter{roemisch} \setcounter{roemisch}{\value{page}}
\clearpage

\setcounter{page}{2} \pagenumbering{arabic}
%-----------------------------------------------------------------------------------------------------------------

\section{Programmstruktur - die wichtigsten Verzeichnisse und Dateien}
\includegraphics{grafiken/ordnerstruktur.jpg}

\subsection{AndroidManifest.xml}
Zu jeder App geh�rt eine zentrale Beschreibungsdatei. Sie enth�lt eine Liste der
Komponenten, aus denen das Programm besteht und befindet sich in der obersten
Ebene des Projektverzeichnisses. Au�erdem werden in ihr die ben�tigten
Berechtigungen sowie etwaige zus�tzlich verwendete Bibliotheken vermerkt. Auch
Angaben zur mindestens n�tigen oder gew�nschten Android-Version werden hier
eingetragen.
\\\includegraphics{grafiken/leereAppAndroidManifest}
\\Die Komponenten einer Anwendung sind Kinder des Elements
\textit{\textless application/\textgreater}. Wenn man im Assistenten zum Anlegen
neuer Projekte Create Activity mit einem H�kchen versieht und einen Namen
eintr�gt, enth�lt das Manifest ein Element \textit{\textless activity
/\textgreater}. Dessen Attribut android:name beinhaltet den im Assistenten
eingegebenen Activity-Namen. Wenn man manuell eine Activity-Klasse anlegt (eine
Klasse anlegt und mit \textit{extends Activity} versieht), muss man nachtr�glich
die erzeugte Activity im Manifest bekannt machen. Mithilfe des Elementes
\textit{\textless intent-filter /\textgreater} kann man eine Activity zur
Haupt-Activity machen. Dessen Kindelement \textit{\textless action
/\textgreater} kennzeichnet die Activity als Haupteinstiegspunkt in die
Anwendung. \textit{\textless category /\textgreater} sorgt daf�r, dass sie im
Programmstarter angezeigt wird.

\subsection{strings.xml}
\includegraphics{grafiken/leereAppStrings.jpg}
\\\includegraphics{grafiken/leereAppStrings1.jpg}
\\Elemente haben einen Titel. �blicherweise werden s�mtliche Titel in die
String-Resource Datei \textit{string.xml} eingetragen und �ber
\textit{@string/\ldots} referenziert. Die Speicherung von Texten an einem
zentralen Ort hat zahlreiche Vorteile. Beispielsweise werden identische
Textteile leichter entdeckt, als wenn diese in den Quelltexten der Klassen
verborgen sind. Damit l�sst sich, wenn auch in eher bescheidenem Umfang,
Speicherplatz sparen. Au�erdem macht die Trennung von Daten und Programmlogik
die Internationalisierung, also die �bersetzung einer App in verschiedene
Sprachen, viel einfacher. Hierzu wird f�r jede zu unterst�tzende Sprache im
Ordner \textit{res} ein Verzeichnis angelegt.
Dessen Name beginnt mit \textit{values-} und endet mit dem ISO-Sprachschl�ssel.
F�r Deutsch ist dies \textit{de}. Das Verzeichnis muss also \textit{values-de}
hei�en. Jeder dieser Ordner erh�lt eine eigene Version von \textit{strings.xml}.
Deren Bezeichner sind stets gleich, die Texte liegen hingegen in den jeweiligen
Sprachen vor. Texte in der Standardsprache verbleiben in \textit{values}.

\subsection{main.xml}
\subsection{activity\_main.xml}


\section{Layouts, Views und Komponenten}
\subsection{Layouts}
\subsection{Views und Widgets}
\subsection{Basiskomponenten einer App}
\subsubsection{Activity}
Normalerweise ist jeder Activity eine Benutzeroberfl�che, also ein Baum bestehend
aus Views und ViewGroups, zugeordnet. Activities bilden demnach die vom Anwender
wahrgenommenen Bausteine einer App.
Sie repr�sentieren also meist die Benutzeroberfl�che und Interaktionen. 
Jede Android- Anwendung besteht deshalb aus mindestens einer Activity.
Activities k�nnen andere Activities aufrufen und mit ihnen Daten austauschen.
Jede Activity besteht aus einer in XML definierten Layout-Datei und einer
dazugeh�rigen Java-Klassendatei, welche bei Android Studio im Verzeichnis
'\textit{\%Appname/src/main/java/\%packagename/}' abgelegt ist. Die dazugeh�rige
Layout-Datei (XML-Datei) ist im Verzeichnis
'\textit{\%Appname/src/main/res/layout}' zu finden.

%onCreate()
%onCreateOptionsmenu
%\ldots

\subsubsection{Intents}
\subsubsection{Fragments}
\subsubsection{Services}

\section{Allgemeiner Ablauf bei einer einfachen Beispiel-App}
\subsection{Anlegen eines Projektes und Erstellung einer MainActivity}
\subsection{Layout festlegen und Views hinzuf�gen}
\subsection{ID und Namen eines jeden Views festlegen}
%string.xml!
\subsection{Einbinden der View-Elemente in die Activity-Klasse}
%findViewById(R.id. ..)


\end{document}
